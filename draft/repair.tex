%%
% Copyright (c) 2017 - 2020, Pascal Wagler;
% Copyright (c) 2014 - 2020, John MacFarlane
%
% All rights reserved.
%
% Redistribution and use in source and binary forms, with or without
% modification, are permitted provided that the following conditions
% are met:
%
% - Redistributions of source code must retain the above copyright
% notice, this list of conditions and the following disclaimer.
%
% - Redistributions in binary form must reproduce the above copyright
% notice, this list of conditions and the following disclaimer in the
% documentation and/or other materials provided with the distribution.
%
% - Neither the name of John MacFarlane nor the names of other
% contributors may be used to endorse or promote products derived
% from this software without specific prior written permission.
%
% THIS SOFTWARE IS PROVIDED BY THE COPYRIGHT HOLDERS AND CONTRIBUTORS
% "AS IS" AND ANY EXPRESS OR IMPLIED WARRANTIES, INCLUDING, BUT NOT
% LIMITED TO, THE IMPLIED WARRANTIES OF MERCHANTABILITY AND FITNESS
% FOR A PARTICULAR PURPOSE ARE DISCLAIMED. IN NO EVENT SHALL THE
% COPYRIGHT OWNER OR CONTRIBUTORS BE LIABLE FOR ANY DIRECT, INDIRECT,
% INCIDENTAL, SPECIAL, EXEMPLARY, OR CONSEQUENTIAL DAMAGES (INCLUDING,
% BUT NOT LIMITED TO, PROCUREMENT OF SUBSTITUTE GOODS OR SERVICES;
% LOSS OF USE, DATA, OR PROFITS; OR BUSINESS INTERRUPTION) HOWEVER
% CAUSED AND ON ANY THEORY OF LIABILITY, WHETHER IN CONTRACT, STRICT
% LIABILITY, OR TORT (INCLUDING NEGLIGENCE OR OTHERWISE) ARISING IN
% ANY WAY OUT OF THE USE OF THIS SOFTWARE, EVEN IF ADVISED OF THE
% POSSIBILITY OF SUCH DAMAGE.
%%

%%
% This is the Eisvogel pandoc LaTeX template.
%
% For usage information and examples visit the official GitHub page:
% https://github.com/Wandmalfarbe/pandoc-latex-template
%%


% @modified: Chuwen <chuwzhang@gmail.com>
% Options for packages loaded elsewhere
\PassOptionsToPackage{unicode}{hyperref}
\PassOptionsToPackage{hyphens}{url}
\PassOptionsToPackage{dvipsnames,svgnames*,x11names*,table}{xcolor}
%
\documentclass[
  a4paper,
,tablecaptionabove
]{scrartcl}
\usepackage{lmodern}
\usepackage{setspace}
\setstretch{1.2}
\usepackage{amssymb,amsmath}
\numberwithin{equation}{section}
\usepackage{ifxetex,ifluatex}
\ifnum 0\ifxetex 1\fi\ifluatex 1\fi=0 % if pdftex
  \usepackage[T1]{fontenc}
  \usepackage[utf8]{inputenc}
  \usepackage{textcomp} % provide euro and other symbols
\else % if luatex or xetex
  \usepackage{unicode-math}
  \defaultfontfeatures{Scale=MatchLowercase}
  \defaultfontfeatures[\rmfamily]{Ligatures=TeX,Scale=1}
\fi
% Use upquote if available, for straight quotes in verbatim environments
\IfFileExists{upquote.sty}{\usepackage{upquote}}{}
\IfFileExists{microtype.sty}{% use microtype if available
  \usepackage[]{microtype}
  \UseMicrotypeSet[protrusion]{basicmath} % disable protrusion for tt fonts
}{}
\makeatletter
\@ifundefined{KOMAClassName}{% if non-KOMA class
  \IfFileExists{parskip.sty}{%
    \usepackage{parskip}
  }{% else
    \setlength{\parindent}{0pt}
    \setlength{\parskip}{6pt plus 2pt minus 1pt}}
}{% if KOMA class
  \KOMAoptions{parskip=half}}
\makeatother
\usepackage{xcolor}
\definecolor{default-linkcolor}{HTML}{A50000}
\definecolor{default-filecolor}{HTML}{A50000}
\definecolor{default-citecolor}{HTML}{4077C0}
\definecolor{default-urlcolor}{HTML}{4077C0}
\IfFileExists{xurl.sty}{\usepackage{xurl}}{} % add URL line breaks if available
\IfFileExists{bookmark.sty}{\usepackage{bookmark}}{\usepackage{hyperref}}
\hypersetup{
  pdfauthor={Chuwen},
  hidelinks,
  breaklinks=true,
  pdfcreator={LaTeX via pandoc with the Eisvogel template}}
\urlstyle{same} % disable monospaced font for URLs
\usepackage[margin=2.5cm,includehead=true,includefoot=true,centering,]{geometry}
% add backlinks to footnote references, cf. https://tex.stackexchange.com/questions/302266/make-footnote-clickable-both-ways
\usepackage{footnotebackref}
\setlength{\emergencystretch}{3em}  % prevent overfull lines
\providecommand{\tightlist}{%
  \setlength{\itemsep}{0pt}\setlength{\parskip}{0pt}}
\setcounter{secnumdepth}{3}

% Make use of float-package and set default placement for figures to H.
% The option H means 'PUT IT HERE' (as  opposed to the standard h option which means 'You may put it here if you like').
\usepackage{float}
\floatplacement{figure}{H}


\usepackage[UTF8, heading=true]{ctex}
\definecolor{tufeijilk}{RGB}{68,87,151}
\hypersetup{colorlinks=true,linkcolor=tufeijilk,urlcolor=cyan}
\newlength{\cslhangindent}
\setlength{\cslhangindent}{1.5em}
\newenvironment{cslreferences}%
  {\setlength{\parindent}{0pt}%
  \everypar{\setlength{\hangindent}{\cslhangindent}}\ignorespaces}%
  {\par}

\author{Chuwen}
\date{\today}



%%
%% added
%%

%
% language specification
%
% If no language is specified, use English as the default main document language.
%

\ifnum 0\ifxetex 1\fi\ifluatex 1\fi=0 % if pdftex
  \usepackage[shorthands=off,main=english]{babel}
\else
  % @update, do not use sfdefault,
  % @chuwen, 20200711
  %   % % Workaround for bug in Polyglossia that breaks `\familydefault` when `\setmainlanguage` is used.
  % % See https://github.com/Wandmalfarbe/pandoc-latex-template/issues/8
  % % See https://github.com/reutenauer/polyglossia/issues/186
  % % See https://github.com/reutenauer/polyglossia/issues/127
  % \renewcommand*\familydefault{\sfdefault}
  %   % load polyglossia as late as possible as it *could* call bidi if RTL lang (e.g. Hebrew or Arabic)
  \usepackage{polyglossia}
  \setmainlanguage[]{english}
\fi



%
% for the background color of the title page
%

%
% break urls
%
\PassOptionsToPackage{hyphens}{url}

%
% When using babel or polyglossia with biblatex, loading csquotes is recommended
% to ensure that quoted texts are typeset according to the rules of your main language.
%
\usepackage{csquotes}

%
% captions
%
\definecolor{caption-color}{HTML}{777777}
\usepackage[font={stretch=1.2}, textfont={color=caption-color}, position=top, skip=4mm, labelfont=bf, singlelinecheck=false, justification=raggedright]{caption}
\setcapindent{0em}

%
% blockquote
%
\definecolor{blockquote-border}{RGB}{221,221,221}
\definecolor{blockquote-text}{RGB}{119,119,119}
\usepackage{mdframed}
\newmdenv[rightline=false,bottomline=false,topline=false,linewidth=3pt,linecolor=blockquote-border,skipabove=\parskip]{customblockquote}
\renewenvironment{quote}{\begin{customblockquote}\list{}{\rightmargin=0em\leftmargin=0em}%
\item\relax\color{blockquote-text}\ignorespaces}{\unskip\unskip\endlist\end{customblockquote}}

%
% heading color
%
\definecolor{heading-color}{RGB}{40,40,40}
\addtokomafont{section}{\color{heading-color}}
% When using the classes report, scrreprt, book,
% scrbook or memoir, uncomment the following line.
%\addtokomafont{chapter}{\color{heading-color}}

%
% variables for title and author
%
\usepackage{titling}
\title{}
\author{Chuwen}

%
% tables
%

%
% remove paragraph indention
%
\setlength{\parindent}{0pt}
\setlength{\parskip}{6pt plus 2pt minus 1pt}
\setlength{\emergencystretch}{3em}  % prevent overfull lines

%
%
% Listings
%
%


%
% header and footer
%
\usepackage{fancyhdr}

\fancypagestyle{eisvogel-header-footer}{
  \fancyhead{}
  \fancyfoot{}
  \lhead[\today]{}
  \chead[]{}
  \rhead[]{\today}
  \lfoot[\thepage]{Chuwen}
  \cfoot[]{}
  \rfoot[Chuwen]{\thepage}
  \renewcommand{\headrulewidth}{0.4pt}
  \renewcommand{\footrulewidth}{0.4pt}
}
\pagestyle{eisvogel-header-footer}

%%
%% end added
%%

\begin{document}

%%
%% begin titlepage
%%

%%
%% end titlepage
%%



{
        \setcounter{tocdepth}{3}
    \tableofcontents
      }
\hypertarget{progress-summary}{%
\section{Progress summary}\label{progress-summary}}

\begin{itemize}
\tightlist
\item
  Polynomial-time algorithm for deterministic problem:

  \begin{itemize}
  \tightlist
  \item
    The problem can be solved with subgradient method in polynomial
    time. If we allow tolerance \(\epsilon \ge 0\) on subgradient
    method, the worst complexity is
    \(O\left(\frac{1}{\epsilon^2}\cdot\tau\cdot|I|\cdot|T|^3\right)\).
  \item
    Converge and rate:

    \begin{itemize}
    \tightlist
    \item
      we can only retrieve a feasible primal solution from the simplest
      subgradient method by Polyak (1967).
    \item
      Good approximation to optimal solution can be found via the volume
      algorithm by Barahona and Anbil (2000).
    \item
      The rate and convergence of such class of algorithm can be found
      in Nesterov (2009), Nedić and Ozdaglar (2009).
    \item
      todo: try alternative formulation:
      \(U^\top e - d + \epsilon_+ - \epsilon_- = 0\)
    \end{itemize}
  \item
    Details:

    \begin{itemize}
    \tightlist
    \item
      The subproblem at each subgradient iteration on dual problem can
      be solved by dynamic programming, at the cost of
      \(O\left(\tau\cdot|T|^3 \right)\).
    \item
      The dual variables can be solved \textbf{analytically}.
    \end{itemize}
  \item
    Computations has been done for DP, subgradient method (volume
    algorithm).
  \end{itemize}
\end{itemize}

a quick look at the results (\(|I| = 10, |T| = 20, 10\) random generated
instances)

\begin{tabular}{lrrrrr}
\toprule
{} &      sg\_lb &     sg\_val &   bench\_lb &  bench\_sol &  primal\_gap \\
\midrule
0 &  14.499735 &  17.533016 &  16.000000 &         16 &    0.095814 \\
1 &  32.631830 &  35.403596 &  31.979908 &         34 &    0.041282 \\
2 &   9.267048 &  12.402154 &   2.999999 &         10 &    0.240215 \\
3 &  52.727507 &  55.299756 &  49.496651 &         54 &    0.024070 \\
4 &   0.000000 &   4.745779 &   0.000000 &          0 &         inf \\
5 &  15.927653 &  16.950710 &  16.000000 &         16 &    0.059419 \\
6 &  50.000000 &  52.318332 &  50.000000 &         50 &    0.046367 \\
7 &   8.000000 &  10.682970 &   8.663961 &         10 &    0.068297 \\
8 &  34.000000 &  35.615792 &  32.916046 &         34 &    0.047523 \\
9 &  24.000000 &  25.193067 &  24.000000 &         24 &    0.049711 \\
\bottomrule
\end{tabular}

\hypertarget{the-air-repair-model}{%
\section{The air-repair model}\label{the-air-repair-model}}

\begin{quote}
Notation
\end{quote}

\begin{itemize}
\tightlist
\item
  \(I, T\) - set of plane, time periods, respectively
\item
  \(b, h\) - demand withdraw and plane idle cost, respectively
\item
  \(\tau\) - lead time for maintenance
\end{itemize}

The demand is stochastic with some distribution \(f\in \mathscr F\)

\begin{itemize}
\tightlist
\item
  \(\mathbfit d_t\) - demand/number of planes needed at time \(t\)
\end{itemize}

\begin{quote}
Decision
\end{quote}

\begin{itemize}
\tightlist
\item
  \(x_{it}\) - 0 - 1 variable, 1 if plane \(i\) starts a maintenance at
  time \(t\)
\item
  \(u_{it}\) - 0 - 1 variable, 1 if plane is working at time \(t\)
\item
  \(s_{it} \ge 0\) - the lifespan of plane \(i\) at time \(t\)
\end{itemize}

The DRO/SP model, the goal is to minimize unsatisfied demand and surplus
(idle) flights, using a newsvendor-like objective function

\[\begin{aligned}
& q_t \equiv b \cdot (d_t - \sum_i u_{it})_+ + h \cdot  ( \sum_i u_{it} - d_t)_+  \\
  & \inf \max_{f\in \mathscr F}\mathbb E_f \left[ \sum_t q_t  \right] \\
  \mathbf{s.t.}  & \\
  & q_t \ge b\cdot \left (d_t - \sum_i u_{it} \right) & \forall t \in T \\
  & q_t \ge h\cdot \left (\sum_i u_{it} - d_t \right ) & \forall t \in T \\
  &  s_{i, t+1} =  s_{i t}  - \alpha_i  u_{it} + \beta_i  x_{i, t- \tau} & \forall i \in I, t \in T\\
  &  x_{it} +  u_{i, t} \le 1& \forall i \in I, t \in T\\
  &  x_{it} + x_{i\rho} + u_{i, \rho} \le 1& \forall i \in I,  t\in T, \rho = t + 1, ..., t+\tau \\
  &   s_{i t} \ge L& \forall i \in I, t \in T 
\end{aligned}\]

We define the last four sets of constraint as \(\Omega_i\), which
describe the non-overlapping requirements during a maintenance.

\hypertarget{deterministic}{%
\section{Deterministic}\label{deterministic}}

We first consider the deterministic problem.

The problem can be solved with subgradient method in polynomial time and
the solution is exact. If we allow tolerance \(\epsilon \ge 0\) on
subgradient method, the complexity is:

\[O\left(\frac{1}{\epsilon^2}\cdot\tau\cdot|I|\cdot|T|^3\right)\]

\hypertarget{subgradient-method}{%
\subsection{Subgradient method}\label{subgradient-method}}

Relax binding constraints of \(q\)

\[\begin{aligned}
  z_{\mathsf{LD}}(\lambda, \mu)& = \inf_{x, s, u} \left[ \sum_t 
      q_t (1-\lambda_t - \theta_t) + d_t(b\lambda_t -h \theta_t)
    + \sum_i \sum_t u_{i,t}(h\theta_t - b\lambda_t) \right] \\ 
  \mathbf {s.t. }  & \\
  & x_{(i,\cdot)}, u_{(i,\cdot)}, s_{(i,\cdot)} \in \Omega_i
\end{aligned}\]

\(\Omega_i\) defined as the region of
\(x_{(i,\cdot)}, u_{(i,\cdot)}, s_{(i,\cdot)}, \forall i\in I\).

we notice \(z_{\mathsf{LD}}(\lambda, \theta)\) is unbounded unless:
\[\lambda, \theta \in \{\lambda, \theta \ge 0 \big | (1-\lambda_t - \theta_t) \ge 0,\forall t\in T\}\]

we have: \[\begin{aligned}
z_{\mathsf{LD}}(\lambda, \theta) = \sum_t 
      d_t(b\lambda_t -h \theta_t) + \inf_{x, s, u} \left[  \sum_i \sum_t u_{i,t}(h\theta_t - b\lambda_t) \right]
\end{aligned}\]

We wish to solve: \[\sup_{(\lambda, \theta)}z_{\mathsf{LD}}\]

We can solve for each \(i\in I\) independently at each iteration of a
subgradient method.

Notice:

\begin{itemize}
\tightlist
\item
  at iteration \(k\), suppose multipliers
  \(\lambda^k + \theta^k \le 1\), if \(( x^\star, s^\star, u^\star)\)
  solves the relaxed minimization problem, then it is also feasible for
  the original problem (compute \(q\) accordingly). The optimality gap
  can easily be calculated by simple evaluations.
\item
  at each iteration \(k\), the sub-gradients: \(\mathcal P\) is the
  orthogonal projection onto
  \(\mathcal D =\left\{(\lambda, \theta) \;| \; \lambda + \theta \le 1\right\}\)
\end{itemize}

\[\begin{aligned}
  &\nabla \lambda^k = b\cdot \left(d -  (U^k)^\top e\right) \\
  &\nabla \theta^k = h\cdot \left((U^k)^\top e - d\right) \\
  &\theta^{k+1} = \mathcal P (\theta^k + a^k\nabla\theta^k ) \\
  &\lambda^{k+1} = \mathcal P (\lambda^k + a^k\nabla\lambda^k )
\end{aligned}\]

\begin{itemize}
\tightlist
\item
  the projection \(\mathcal P\) can be computed very easily. Since the
  projection \((\tilde\lambda, \tilde\theta)\) onto \(\mathcal D\) can
  be formulated as the model
  \(\inf_{\lambda\ge 0, \theta \ge 0, \lambda + \theta \le e} ||\tilde\lambda-\lambda||^2 + ||\tilde\theta - \theta||^2\),
  and solved analytically.
\end{itemize}

\hypertarget{subproblem-for-each-plane}{%
\subsection{Subproblem for each plane}\label{subproblem-for-each-plane}}

The subproblem \(\forall i\in I\) is defined as follows:

\[\begin{aligned}
c_t \equiv (h\theta_t - b\lambda_t) \\
\inf_{\Omega_i} \sum_t c_t \cdot u_{i,t}
\end{aligned}\]

The model describes a problem to maximize total utility while keeping
the lifespan safely away from the lower bound \(L\). Define state:
\(y_t = \left[m_t,s_t \right]^\top\), where \(m_t\) \textbf{denotes the
remaining time of the undergoing maintenance}. \(s_t\) is the remaining
lifespan.

At each period \(t\) we decide whether the plane \(i\) is idle or
waiting (for the maintenance), working, or starting a maintenance, i.e.:

\[(u_t, x_t) \in \left\{(1, 0), (0,0), (0, 1)\right\}\]

We have the optimal equation:
\[V_n(u_t, x_t | m_t, s_t) = c_t \cdot u_t + \inf_{u,x} V_{n-1}(...)\]

Complexity: let \(s_0\) be the initial lifespan and finite time horizon
be \(|T|\), we notice the states for remaining maintenance waiting time
is finite, \(m_t \in \{0, 1, ..., \tau\}\).

Let total number of possible periods to initiate a maintenance be
\(n_1\), and working periods be \(n_2\). If we ignore lower bound \(L\)
on \(s\), total number of possible values of \(s\) is bounded above:
\(|s| = \sum_i^{|T|}\sum_j^{|T| - i} 1=(|T| + 1)(\frac{1}{2}|T| + 1)\)
since \(n_1 + n_2 \le |T|\). For each subproblem we have at most 3
actions, thus we conclude this problem can be solved by dynamic
programming in polynomial time, the complexity is:
\(O\left(\tau\cdot|T|^3 \right)\)

\hypertarget{stochastic}{%
\section{Stochastic}\label{stochastic}}

We use boldface notation to denote random variables. We let the vector
\(\mathbfit q = [\mathbfit q_1, ..., \mathbfit q_{|T|}]^\top, \mathbfit d = [\mathbfit d_1,...,\mathbfit d_{|T|}]^\top\),
\(\Xi_d\) be the support for \(\mathbfit{d}\). For simplicity, we let
\(\mathbfit e\) be the vector of ones of corresponding dimension in
matrix-vector calculations.

The DRO/SP model, the goal is to minimize worst-case expected
unsatisfied demand and surplus (idle) flights

\[\begin{aligned}
  & \inf \sup_{f\in \mathscr F}\mathbb E_f \left[ \sum_t \mathbfit q_t  \right] \\
  \mathbf{s.t.}  & \\
  & \mathbfit q \ge b\cdot \left (\mathbfit d - \mathbfit U^\top \mathbfit{e} \right) & \forall \mathbfit d \in \Xi_d\\
  & \mathbfit q \ge h\cdot \left ( \mathbfit U^\top \mathbfit{e}  - \mathbfit d \right ) & \forall \mathbfit d \in \Xi_d \\
  & \mathbfit x_{(i,\cdot)}, \mathbfit u_{(i,\cdot)}, \mathbfit s_{(i,\cdot)} \in \Omega_i & \forall i\in I
\end{aligned}\]

Same relaxation scheme can be used on the DRO models:

\begin{itemize}
\tightlist
\item
  Mean-variance, in Delage and Ye (2010).
\item
  Likelihood, in Wang et al. (2016)
\end{itemize}

\hypertarget{mean-variance}{%
\subsection{Mean-variance}\label{mean-variance}}

With moment uncertainty for
\(\mathbfit d: \mathbb{E}(\mathbfit d) = \mu_0, ...\), in Delage and
Ye (2010). The DRO model is equivalent to the following problem:

\[\begin{aligned}
\inf_{\mathbfit{x}, \mathbfit{Q}, \mathbfit{\beta}, r, s} & \left(\gamma_{2} \mathbfit{\Sigma}_{0}-\mathbfit{\mu}_{0} \mathbfit{\mu}_{0}^{\top}\right) \bullet \mathbfit{Q}+r+\left(\mathbfit{\Sigma}_{0} \bullet \mathbfit{P}\right)-2 \mathbfit{\mu}_{0}^{\top} \mathbfit{p} + \gamma_{1} s \\
\mathbf { s.t. } & \\
& \mathbfit{d}^{\top} \mathbfit{Q} \mathbfit{d} -\mathbfit{d^\top\beta} + r \ge \sum_t \mathbfit q_t & \forall \mathbfit d \in \Xi_d \\
& \mathbfit{Q} \succeq 0, \mathbfit{\beta} \in \mathbb{R}^{|T|}, 
\begin{bmatrix}
  \mathbfit{P} & \mathbfit{p} \\
  \mathbfit{p}^\top & s
\end{bmatrix} \succeq 0, \; 
\mathbfit \beta = 2 (\mathbfit p + \mathbfit{Q\mu}_0)\\
& \mathbfit q \ge b\cdot \left (\mathbfit d - \mathbfit U^\top \mathbfit{e} \right) & \forall \mathbfit d \in \Xi_d\\
  & \mathbfit q \ge h\cdot \left ( \mathbfit U^\top \mathbfit{e}  - \mathbfit d \right ) & \forall \mathbfit d \in \Xi_d \\
& \mathbfit x_{(i,\cdot)}, \mathbfit u_{(i,\cdot)}, \mathbfit s_{(i,\cdot)} \in \Omega_i & \forall i\in I
\end{aligned}\]

\hypertarget{finite-support-and-likelihood-robust}{%
\subsection{Finite support and likelihood
robust}\label{finite-support-and-likelihood-robust}}

The problem is, let
\(\mathbfit Q = [\mathbfit{q}^1, ..., \mathbfit{q}^N]\)

\[\begin{aligned}
  & \sup_{\beta, \theta, \Omega_i, \forall i} \theta + \beta \gamma +  \beta N - \underbrace{\beta \mathbf N^\top \log(\frac{\beta \mathbf N}{\mathbfit{Q} e-\theta \mathbf 1})}_{\mathcal D_{KL}(\beta \mathbf N | \mathbfit{Q} e-\theta \mathbf 1)}  \\
  \textbf {s.t.} \\
  & \beta \ge 0 \\
  & \mathbfit{Q} e \ge \theta \mathbf 1 \\ 
  & \mathbfit q^n \ge b\cdot \left (\mathbfit d^n - \mathbfit{U}^\top e \right) & \forall n = 1, ..., N\\
  & \mathbfit q^n \ge h\cdot \left ( \mathbfit{U}^\top e  -\mathbfit d^n  \right ) & \forall n = 1, ..., N \\
  & \mathbfit x_{(i,\cdot)}, \mathbfit u_{(i,\cdot)}, \mathbfit s_{(i,\cdot)} \in \Omega_i & \forall i\in I
\end{aligned}\]

Relax binding constraints for \(\mathbfit Q, \mathbfit U\):
\[\mathbfit z_{\mathsf{LD}} = \sup_{\mathbfit Q} ...+ \sum_n (b\lambda^n - h\theta^n)^\top \mathbfit{d}^n + \sum_n(\lambda^n +\theta^n)^\top q^n  \\
+ \sup_{\mathbfit U}\sum_i\sum_t(\sum_nh\theta^n_t - b\lambda^n_t)\mathbfit u_{it}\]

We can optimize for \(\mathbfit Q, \mathbfit U\) separately, at each
step, subproblems can be solved in polynomial time.

\hypertarget{reference}{%
\section*{Reference}\label{reference}}
\addcontentsline{toc}{section}{Reference}

\hypertarget{refs}{}
\begin{cslreferences}
\leavevmode\hypertarget{ref-barahona_volume_2000}{}%
Barahona F, Anbil R (2000) The volume algorithm: Producing primal
solutions with a subgradient method. \emph{Mathematical Programming}
87(3):385--399.

\leavevmode\hypertarget{ref-delage_distributionally_2010}{}%
Delage E, Ye Y (2010) Distributionally robust optimization under moment
uncertainty with application to data-driven problems. \emph{Operations
Research} 58(3).

\leavevmode\hypertarget{ref-nedic_approximate_2009}{}%
Nedić A, Ozdaglar A (2009) Approximate primal solutions and rate
analysis for dual subgradient methods. \emph{SIAM Journal on
Optimization} 19(4):1757--1780.

\leavevmode\hypertarget{ref-nesterov_primal-dual_2009}{}%
Nesterov Y (2009) Primal-dual subgradient methods for convex problems.
\emph{Mathematical programming} 120(1):221--259.

\leavevmode\hypertarget{ref-polyak_general_nodate}{}%
Polyak BT (1967) A general method for solving extremal problems.
\emph{Soviet Mathematics Doklady}:5.

\leavevmode\hypertarget{ref-wang_likelihood_2016}{}%
Wang Z, Glynn PW, Ye Y (2016) Likelihood robust optimization for
data-driven problems. \emph{Computational Management Science}
13(2):241--261.
\end{cslreferences}

\end{document}

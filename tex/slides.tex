\documentclass{beamer}

\usepackage[font=small,labelfont=bf]{caption}
\usepackage{longtable}
\usepackage{subfiles}
\usepackage{subfig}
\usepackage{booktabs}
\setlength{\tabcolsep}{6pt}
% \usepackage[table]{xcolor}
\title{Dual Optimization for Newsvendor-like Problem}
% \title{Dual Method for Solving Flight Maintenance Scheduling Problem}
\author{Chuwen}
\date{\today}

\begin{document}
\frame{\titlepage}


\begin{frame}
  \frametitle{Primal problem}

  \begin{align}
                                  & \min f(\delta, \epsilon)                                                                       \\
    \label{newsvendor} s.t. \quad & y + \delta - \epsilon = b                                                                      \\
                                  & y \in \Omega_y \subseteq \mathbb{R}^n, \delta \in \mathbb{R}^n_+ , \epsilon \in \mathbb{R}^n_+
  \end{align}

  \begin{itemize}
    \item Assume \(f=p^\mathsf{T}\delta + h^\mathsf{T}\epsilon, p \ge 0, h \ge 0\), \eqref{newsvendor} expresses the newsvendor objective.
    \item \(\Omega_y\) is a mixed integer set, and can be decomposed into small problems that are easier to solve.
  \end{itemize}
\end{frame}


\begin{frame}
  \frametitle{Lagrangian relaxation}

  Relax the newsvendor equation: \(y + \delta - \epsilon = b\)

  \subtitle{Dual function}
  \begin{equation}\label{eq:dual}
    \begin{aligned}
      \phi(\lambda) = & \min_{\delta, \epsilon} (p + \lambda)^\mathsf{T}\delta + (h - \lambda)^\mathsf{T}\epsilon + \min_y \lambda^\mathsf{T} y - \lambda^\mathsf{T} b \\
      =               & \min_y \lambda^\mathsf{T} y - \lambda^\mathsf{T} b                                                                                             \\
      \mathbf{s.t.}   &                                                                                                                                                \\
                      & y \in \Omega_y                                                                                                                                 \\
                      & \delta \in \mathbb{R}^n_+ , \epsilon \in \mathbb{R}^n_+
    \end{aligned}
  \end{equation}
  (Since \(\lambda \in [-p, h] \) and \(\delta^\star = \epsilon^\star = 0\) else unbounded)

\end{frame}


\begin{frame}
  \frametitle{Subgradient method}
  We want to solve \(\max_\lambda \phi(\lambda)\) by subgradient method:
  \begin{equation}\label{eq:simple_subgrad}
    \begin{aligned}
       & g = y - b  \in \partial \phi                                     \\
       & g_k = y_k - b                                                    \\
       & \lambda_{k+1} = \mathbf{P}(\lambda_{k} + s_{k}g_{k})             \\
       & s_{k} = \gamma_k\frac{\phi^\star - \phi(\lambda_k)}{\|g_{k}\|^2}
    \end{aligned}
  \end{equation}
  \(\mathbf{P}\) is the projection onto \([-p, h]\). Keep the averaged solution:
  \begin{equation}\label{eq:avg}
    \bar y_k = \frac{1}{k}\sum_i^k y_i
  \end{equation}

\end{frame}
\begin{frame}
  \frametitle{Primal recovery}
  \((y_k, \epsilon_k = 0, \delta_k = 0)\) may not be feasible, use recovery:
  \begin{equation}\label{eq:recovery}
    \begin{aligned}
       & \epsilon_k = \max\{y_k - b, 0\}           \\
       & \delta_k = \max\{b - y_k, 0\}             \\
       & \bar \epsilon_k = \max\{\bar y_k - b, 0\} \\
       & \bar \delta_k = \max\{b - \bar y_k, 0\}
    \end{aligned}\end{equation}

  let corresponding primal value  be \(z(y_k) = f(\delta_k, \epsilon_k)\), \(\bar z_k = z(\bar y_k)\)
\end{frame}

\begin{frame}
  \frametitle{Motivation: fleet engine maintenance problem (FMP)}
  \begin{itemize}
    \item engines: \(i \in I\), time periods: \(t = 1, 2, ..., n\), demand: \(b = (b_1, ..., b_n)\).
    \item at each time we decide if engine \(i\) is working \(u_{it} = 1\) or sent to maintenance \(x_{it} = 1\) (and will be finished after \(\tau\) periods)
    \item the lifespan of engine \(i\) decreases by \(\alpha_i\) if working; increases by \(\beta_i\) if the maintenance is finished; the lifespan has a lower bound \(L\).
    \item our goal is to satisfy demand \(b\) by minimizing the surplus \(\epsilon\) and shortage \(\delta\): \(f=p^\mathsf{T}\delta + h^\mathsf{T}\epsilon\)
    \item let \(\Omega_i\) be the mixed-integer set regarding the maintenance requirements individually, so we have \(\Omega_i\) for each \(i\)
  \end{itemize}
\end{frame}

\begin{frame}
  \frametitle{FMP}
  \begin{align}
    \label{eq:fmp.obj}  f =    & \min_{x_{it}, u_{it}, \delta_t, \epsilon_t} \sum_t (b\cdot  \delta_t + h \cdot \epsilon_t)                                                            \\
    \nonumber \mathbf{s.t.}    &                                                                                                                                                       \\
    \label{eq:fmp.demand}      & \sum_i u_{it} + \delta_t - \epsilon_t = d_t                                                ,\quad\forall t \in T                                      \\
    \label{eq:fmp.lifespan}    & s_{i, t+1} =  s_{i t}  - \alpha_i  u_{it} + \beta_i  x_{i, t- \tau}                        ,\quad\forall i \in I, t \in T                             \\
    \label{eq:fmp.work_or_not} & x_{it} +  u_{i, t} \le 1                                                                  ,\quad \forall i \in I, t \in T                             \\
    \label{eq:fmp.leadtime}    & x_{it} + x_{i\rho} + u_{i, \rho} \le 1                                                     ,\quad \forall i \in I,  t\in T, \rho = t + 1, ..., t+\tau \\
    \label{eq:fmp.lifespan_lb} & s_{i t} \ge L                                                                              ,\quad \forall i \in I, t \in T
  \end{align}
  \eqref{eq:fmp.lifespan} - \eqref{eq:fmp.lifespan_lb} can be expressed as \(\Omega_i, \forall i \in I\)
\end{frame}

\begin{frame}
  \frametitle{FMP: continued}
  \begin{itemize}
    \item \eqref{eq:fmp.demand} is the demand satisfaction constraint.
    \item \eqref{eq:fmp.lifespan} tracks the engine lifespan.
    \item \eqref{eq:fmp.work_or_not} means an engine cannot work if sent to maintenance.
    \item \eqref{eq:fmp.leadtime} means the maintenance must be finished before an engine does anything else.
    \item \eqref{eq:fmp.lifespan_lb} denotes the lower bound of lifespan.
    \item if relax \eqref{eq:fmp.demand} then we can solve for each \(i\) individually.
  \end{itemize}
\end{frame}

\begin{frame}
  \frametitle{FMP: Lagrangian relaxation}
  For the FMP, dual function:
  \[\begin{aligned}
      \phi(\lambda) & = - \sum_t \lambda_t d_t + \sum_i \min_{\Omega_i} \sum_t\lambda_t u_{it} \\
    \end{aligned}\]

  It reduces to a set of low dimensional minimization problems for each \(i \in I, \forall \lambda\)

  (Recall \(\lambda_t \in [-b, h] \) and \(\delta_t^\star = \epsilon_t^\star = 0\) else unbounded)
  \begin{equation}\label{subproblem}\begin{aligned}
      \min_{\Omega_i} \sum_t \lambda_t \cdot u_{i,t}
    \end{aligned}\end{equation}
  \eqref{subproblem} is the subproblem to be solved by dynamic programming. (states: lifespan, action: work or start maintenance)
\end{frame}

\begin{frame}
  \frametitle{FMP: subgradient method}
  At each iteration \(k\):
  \begin{itemize}
    \item  \(y_k\) solves \(\phi(\lambda_k) = \min_y \lambda_k^\mathsf{T}(y-b)\)
    \item for FMP: \(y_k = U_k ^\mathsf{T} 1 - d, U_k = (u^{(k)}_{it})\), solved from \eqref{subproblem} by DP.
    \item update \(\lambda_k\) by \eqref{eq:simple_subgrad}
  \end{itemize}
\end{frame}
\begin{frame}
  \frametitle{Results}

  Tests are done on FMP with random instances: rand(low, high) means random integer in [low, high]

  \begin{itemize}
    \item L = 2
    \item \(d_t= \textsf{rand}(\frac{|I|}{2}, |I|), \forall t\)
    \item \(\tau_i = \textsf{rand}(2, 5), a_i =  \textsf{rand}(2, 5),  b_i =  \textsf{rand}(5, 10), \forall i\)
    \item \(s_{i, 0} = \textsf{rand}(5, 8), \forall i\)
  \end{itemize}

  We consider two cases for objective function
  \begin{itemize}
    \item [i], time-invariant: \(h_t = 11, p_t = 18, \forall t\)
    \item [ii], \(h_t = \textsf{rand}(10, 15), p_t = \textsf{rand}(10, 16), \forall t\)
  \end{itemize}
\end{frame}

\begin{frame}
  \frametitle{Results}

  The subgradient method (sg) stops at (maximum) 400 iterations, then compared with benchmarks by Gurobi (grb).
  \begin{itemize}
    \item lb\_grb, \(f_{\textsf{grb}}\)  are lower bound, primal value from grb.
    \item t\_grb, t\_sg are runtime from grb and sg, respectively.
    \item \(\phi_{\textsf{sg}}\) is the dual value (\(\phi\)) by sg.
    \item \(z_{\textsf{sg}}\) is primal value in sg using averaged primal recovery, i.e.
          \[z_{\textsf{sg}} = z(\bar y_k) = f(\bar \delta_k, \bar \epsilon_k)\]
    \item \(\phi_{\textsf{gap}}\), z\_gap are relative gap from sg to grb for dual and primal values.
  \end{itemize}
\end{frame}

\begin{frame}
  \frametitle{FMP: invariant \(h_t = h, b_t = b, \forall t = 1, ..., n\)}
  \scriptsize
  \begin{tabular}{lllrrrlllll}
    \toprule
    {} & lb\_grb              & \(f_{\textsf{grb}}\)   & t\_grb
       & t\_sg                & \(\phi_{\textsf{sg}}\) & \(\phi_{\textsf{gap}}\)
       & \( z_{\textsf{sg}}\) & z\_gap                                                                                          \\
    \midrule
    0  & 936.00               & 936.00                 & 23.93                   & 11.46 & 936.00  & -0.00\% & 952.02  & 1.71\% \\
    1  & 954.00               & 954.00                 & 11.00                   & 8.88  & 954.00  & 0.00\%  & 958.78  & 0.50\% \\
    2  & 1224.00              & 1224.00                & 1.11                    & 6.74  & 1224.00 & 0.00\%  & 1230.12 & 0.50\% \\
    3  & 1026.00              & 1026.00                & 2.76                    & 7.58  & 1026.00 & 0.00\%  & 1031.13 & 0.50\% \\
    4  & 882.00               & 882.00                 & 3.41                    & 6.98  & 882.00  & 0.00\%  & 886.39  & 0.50\% \\
    5  & 990.00               & 990.00                 & 0.94                    & 9.35  & 990.00  & 0.00\%  & 994.95  & 0.50\% \\
    6  & 774.00               & 774.00                 & 1.08                    & 9.73  & 774.00  & 0.00\%  & 778.58  & 0.59\% \\
    7  & 864.00               & 864.00                 & 1.26                    & 9.06  & 864.00  & 0.00\%  & 868.34  & 0.50\% \\
    8  & 972.00               & 972.00                 & 1.02                    & 6.97  & 972.00  & 0.00\%  & 976.85  & 0.50\% \\
    9  & 648.00               & 648.00                 & 88.56                   & 16.14 & 645.14  & -0.44\% & 665.45  & 2.69\% \\
    \bottomrule
  \end{tabular}
  \normalsize
\end{frame}

\begin{frame}
  \frametitle{FMP: variant \(h, b\)}
  \scriptsize
  \begin{tabular}{lllrrrlllll}
    \toprule
    {} & lb\_grb              & \(f_{\textsf{grb}}\)   & t\_grb
       & t\_sg                & \(\phi_{\textsf{sg}}\) & \(\phi_{\textsf{gap}}\)
       & \( z_{\textsf{sg}}\) & z\_gap                                                                                         \\
    \midrule
    0  & 608.00               & 608.00                 & 8.34                    & 12.76 & 607.57 & -0.07\% & 650.63 & 7.01\%  \\
    1  & 587.00               & 587.00                 & 3.93                    & 12.01 & 586.60 & -0.07\% & 619.44 & 5.53\%  \\
    2  & 538.00               & 538.00                 & 22.73                   & 12.33 & 537.36 & -0.12\% & 574.96 & 6.87\%  \\
    3  & 534.00               & 534.00                 & 4.05                    & 13.03 & 532.69 & -0.24\% & 607.16 & 13.70\% \\
    4  & 684.00               & 684.00                 & 19.14                   & 7.61  & 684.00 & 0.00\%  & 691.68 & 1.12\%  \\
    5  & 620.00               & 620.00                 & 4.88                    & 8.06  & 619.98 & -0.00\% & 625.29 & 0.85\%  \\
    6  & 657.00               & 657.00                 & 4.47                    & 9.70  & 655.73 & -0.19\% & 682.15 & 3.83\%  \\
    7  & 625.00               & 625.00                 & 8.25                    & 12.04 & 622.30 & -0.43\% & 670.87 & 7.34\%  \\
    8  & 422.00               & 422.00                 & 5.03                    & 12.40 & 421.25 & -0.18\% & 453.72 & 7.52\%  \\
    9  & 485.00               & 485.00                 & 4.32                    & 13.59 & 484.87 & -0.03\% & 517.35 & 6.67\%  \\
    \bottomrule
  \end{tabular}
  \normalsize
\end{frame}

\begin{frame}
  \frametitle{Conclusion}

  \begin{itemize}
    \item zero duality gap: \(\phi^\star = f^\star\), \(\phi^\star\) is the best bound by \(\phi\) and \( f^\star\) is the best primal value.
    \item can we bound the quality of heuristic for averaged solution? \(\bar z_k = z(\bar y_k)\) converges to \(\phi^\star\):
          \[|\bar z_k - \phi^\star| \]
    \item for variant case, can we improve \(\bar z = z(\bar y)\)?
  \end{itemize}
\end{frame}

\end{document}

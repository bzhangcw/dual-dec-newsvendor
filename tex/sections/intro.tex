\documentclass[../main]{subfiles}

\begin{document}
\maketitle
{
  \setcounter{tocdepth}{3}
  \tableofcontents
}


\section{Introduction}\label{introduction}

This paper is concerned with minimizing a newsvendor-like objective \(f: \mathbb R^n \to \mathbb R\),

\begin{equation}\label{eq:primal}
  \begin{aligned}
                  & \min f(\delta, \epsilon)                                                                       \\
    \mathbf{s.t.} &                                                                                                \\
                  & y + \delta - \epsilon = b                                                                      \\
                  & y \in \Omega_y \subseteq \mathbb{R}^n, \delta \in \mathbb{R}^n_+ , \epsilon \in \mathbb{R}^n_+
  \end{aligned}
\end{equation}

where \(f\) is a convex function of \(\delta, \epsilon\). The
right-hand-side on the binding constraints is in the positive orthant:
\(b \in \mathbb R_+^n\).  In the basic
settings, let \(y\) be the ordering quantity quantities in a multi-item multi-period
newsvendor problem, one minimizes the total expected cost:

\[\min_{y \in \mathbb R_+} \mathbf E\left(h\cdot e^\mathsf{T} \max\{y - b,  0\} + p \cdot e^\mathsf{T} \max\{b - y,  0\}\right)\]

Once the expectation operator is dropped, it is easy to verify the equivalence of such deterministic version
to the problem \eqref{eq:primal} above. This problem is motivated from applications
in device maintenance, inventory management, crew scheduling and so on.



Let \(\lambda\in\mathbb{R}^n\) be the Lagrangian multiplier, we have the Lagrangian dual function,

\begin{equation}\label{eq:dual}
  \begin{aligned}
    \phi(\lambda) = & \min_{\delta, \epsilon} f(\delta, \epsilon) + \lambda^\mathsf{T}\delta - \lambda^\mathsf{T} \epsilon+ \min_y \lambda^\mathsf{T} y - \lambda^\mathsf{T} b \\
    \mathbf{s.t.}   &                                                                                                                                                          \\
                    & y \in \Omega_y                                                                                                                                           \\
                    & \delta \in \mathbb{R}^n_+ , \epsilon \in \mathbb{R}^n_+
  \end{aligned}
\end{equation}

with two independent subproblems.
For \(\delta, \epsilon\) we have a convex optimization problem in the positive orthant.
We also assume minimizing the linear objective under \(y\in \Omega_y\) can be well-solved.
In later sections we show some special cases where \(\Omega_y\) may be further decomposed into smaller problems.

Denote $f^\star, \phi^\star$ be the optimal objective for primal and dual problem, respectively.


\subsection{Affine case}\label{affine-case}

Let \(f=p^\mathsf{T}\delta + h^\mathsf{T} \epsilon, p, h \in \mathbb R_+^n\), we have

\[\phi(\lambda) = \min_{\delta, \epsilon} (p+ \lambda)^\mathsf{T}\delta + (h - \lambda)^\mathsf{T} \epsilon+ \min_y \lambda^\mathsf{T} y - \lambda^\mathsf{T} b\]

Then \(\phi\) is unbounded unless \(\lambda \in \Lambda\) where
\(\Lambda = \{\lambda: \lambda \in [-p, h]\}\), in which case

\[\phi(\lambda) = \min_{y\in \Omega_y} \lambda^\mathsf{T} y - \lambda^\mathsf{T} b,\; \lambda\in \Lambda\]

and \(\delta^\star(\lambda), \epsilon^\star(\lambda)= 0\) are corresponding optimizers
for any \(\lambda \in \Lambda\)

\subsection{Conditions for strong
  duality}\label{conditions-for-strong-duality}

It's well known that strong duality does not hold in general. We review
some of the cases here. The Lagrangian duality theory can be found in
any standard text.

\begin{theorem}
  if \(\Omega_y\) is convex then the strong duality holds ...,
  i.e.~\(\phi^\star = f^\star\)
\end{theorem}

add justifications here (slater, ...)

A more interesting result is devoted to mixed integer problems.
We know Lagrangian relaxation produces a bound up to
linear relaxation of a problem with the "easy" constraints
and the convex hull of relaxed constraints.

\textbf{(Review Here)}.

\begin{lemma}
  if \(\Omega_y = \{y \in \mathbb R^n: y \in \Omega, y\in \mathbb Z^n\}\).
  Then we have the following relation for dual function,
  \[ \phi^\star = \min_{\delta, \epsilon} f(\delta, \epsilon)\quad \textbf{ s.t. }  y + \delta - \epsilon = b,\; y \in \textrm{conv}(\Omega_y)\]
\end{lemma}

This immediately allows us to have strong duality by definition of perfect formulation,
in which case the linear relaxation solves the original problem.

\begin{corollary}\label{strong-ip}
  We conclude the strong duality holds since
  \(Y = \{(y, \delta, \epsilon): y + \delta - \epsilon = b,\; y \in \textsf{conv}(\Omega_y)\}\)
  is already \emph{a perfect formulation} in the sense that
  \(Y = \textsf{conv}(Y)\)
\end{corollary}

\textbf{show this or add more conditions to
  justify}

\end{document}

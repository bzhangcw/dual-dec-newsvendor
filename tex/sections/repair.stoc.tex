%%
% Copyright (c) 2017 - 2020, Pascal Wagler;
% Copyright (c) 2014 - 2020, John MacFarlane
%
% All rights reserved.
%
% Redistribution and use in source and binary forms, with or without
% modification, are permitted provided that the following conditions
% are met:
%
% - Redistributions of source code must retain the above copyright
% notice, this list of conditions and the following disclaimer.
%
% - Redistributions in binary form must reproduce the above copyright
% notice, this list of conditions and the following disclaimer in the
% documentation and/or other materials provided with the distribution.
%
% - Neither the name of John MacFarlane nor the names of other
% contributors may be used to endorse or promote products derived
% from this software without specific prior written permission.
%
% THIS SOFTWARE IS PROVIDED BY THE COPYRIGHT HOLDERS AND CONTRIBUTORS
% "AS IS" AND ANY EXPRESS OR IMPLIED WARRANTIES, INCLUDING, BUT NOT
% LIMITED TO, THE IMPLIED WARRANTIES OF MERCHANTABILITY AND FITNESS
% FOR A PARTICULAR PURPOSE ARE DISCLAIMED. IN NO EVENT SHALL THE
% COPYRIGHT OWNER OR CONTRIBUTORS BE LIABLE FOR ANY DIRECT, INDIRECT,
% INCIDENTAL, SPECIAL, EXEMPLARY, OR CONSEQUENTIAL DAMAGES (INCLUDING,
% BUT NOT LIMITED TO, PROCUREMENT OF SUBSTITUTE GOODS OR SERVICES;
% LOSS OF USE, DATA, OR PROFITS; OR BUSINESS INTERRUPTION) HOWEVER
% CAUSED AND ON ANY THEORY OF LIABILITY, WHETHER IN CONTRACT, STRICT
% LIABILITY, OR TORT (INCLUDING NEGLIGENCE OR OTHERWISE) ARISING IN
% ANY WAY OUT OF THE USE OF THIS SOFTWARE, EVEN IF ADVISED OF THE
% POSSIBILITY OF SUCH DAMAGE.
%%

%%
% This is the Eisvogel pandoc LaTeX template.
%
% For usage information and examples visit the official GitHub page:
% https://github.com/Wandmalfarbe/pandoc-latex-template
%%


% @modified: Chuwen <chuwzhang@gmail.com>
% Options for packages loaded elsewhere
\PassOptionsToPackage{unicode}{hyperref}
\PassOptionsToPackage{hyphens}{url}
\PassOptionsToPackage{dvipsnames,svgnames*,x11names*,table}{xcolor}
%
\documentclass[
  a4paper,
,tablecaptionabove
]{scrartcl}
\usepackage{lmodern}
\usepackage{setspace}
\setstretch{1.2}
\usepackage{amssymb,amsmath}
\numberwithin{equation}{section}
\usepackage{ifxetex,ifluatex}
\ifnum 0\ifxetex 1\fi\ifluatex 1\fi=0 % if pdftex
  \usepackage[T1]{fontenc}
  \usepackage[utf8]{inputenc}
  \usepackage{textcomp} % provide euro and other symbols
\else % if luatex or xetex
  \usepackage{unicode-math}
  \defaultfontfeatures{Scale=MatchLowercase}
  \defaultfontfeatures[\rmfamily]{Ligatures=TeX,Scale=1}
\fi
% Use upquote if available, for straight quotes in verbatim environments
\IfFileExists{upquote.sty}{\usepackage{upquote}}{}
\IfFileExists{microtype.sty}{% use microtype if available
  \usepackage[]{microtype}
  \UseMicrotypeSet[protrusion]{basicmath} % disable protrusion for tt fonts
}{}
\makeatletter
\@ifundefined{KOMAClassName}{% if non-KOMA class
  \IfFileExists{parskip.sty}{%
    \usepackage{parskip}
  }{% else
    \setlength{\parindent}{0pt}
    \setlength{\parskip}{6pt plus 2pt minus 1pt}}
}{% if KOMA class
  \KOMAoptions{parskip=half}}
\makeatother
\usepackage{xcolor}
\definecolor{default-linkcolor}{HTML}{A50000}
\definecolor{default-filecolor}{HTML}{A50000}
\definecolor{default-citecolor}{HTML}{4077C0}
\definecolor{default-urlcolor}{HTML}{4077C0}
\IfFileExists{xurl.sty}{\usepackage{xurl}}{} % add URL line breaks if available
\IfFileExists{bookmark.sty}{\usepackage{bookmark}}{\usepackage{hyperref}}
\hypersetup{
  pdfauthor={Chuwen},
  hidelinks,
  breaklinks=true,
  pdfcreator={LaTeX via pandoc with the Eisvogel template}}
\urlstyle{same} % disable monospaced font for URLs
\usepackage[margin=2.5cm,includehead=true,includefoot=true,centering,]{geometry}
% add backlinks to footnote references, cf. https://tex.stackexchange.com/questions/302266/make-footnote-clickable-both-ways
\usepackage{footnotebackref}
\setlength{\emergencystretch}{3em}  % prevent overfull lines
\providecommand{\tightlist}{%
  \setlength{\itemsep}{0pt}\setlength{\parskip}{0pt}}
\setcounter{secnumdepth}{3}

% Make use of float-package and set default placement for figures to H.
% The option H means 'PUT IT HERE' (as  opposed to the standard h option which means 'You may put it here if you like').
\usepackage{float}
\floatplacement{figure}{H}


\usepackage[UTF8, heading=true]{ctex}
\usepackage{booktabs}
\definecolor{tufeijilk}{RGB}{68,87,151}
\hypersetup{colorlinks=true,linkcolor=tufeijilk,urlcolor=cyan}
\newlength{\cslhangindent}
\setlength{\cslhangindent}{1.5em}
\newenvironment{cslreferences}%
  {\setlength{\parindent}{0pt}%
  \everypar{\setlength{\hangindent}{\cslhangindent}}\ignorespaces}%
  {\par}

\author{Chuwen}
\date{\today}



%%
%% added
%%

%
% language specification
%
% If no language is specified, use English as the default main document language.
%

\ifnum 0\ifxetex 1\fi\ifluatex 1\fi=0 % if pdftex
  \usepackage[shorthands=off,main=english]{babel}
\else
  % @update, do not use sfdefault,
  % @chuwen, 20200711
  %   % % Workaround for bug in Polyglossia that breaks `\familydefault` when `\setmainlanguage` is used.
  % % See https://github.com/Wandmalfarbe/pandoc-latex-template/issues/8
  % % See https://github.com/reutenauer/polyglossia/issues/186
  % % See https://github.com/reutenauer/polyglossia/issues/127
  % \renewcommand*\familydefault{\sfdefault}
  %   % load polyglossia as late as possible as it *could* call bidi if RTL lang (e.g. Hebrew or Arabic)
  \usepackage{polyglossia}
  \setmainlanguage[]{english}
\fi



%
% for the background color of the title page
%

%
% break urls
%
\PassOptionsToPackage{hyphens}{url}

%
% When using babel or polyglossia with biblatex, loading csquotes is recommended
% to ensure that quoted texts are typeset according to the rules of your main language.
%
\usepackage{csquotes}

%
% captions
%
\definecolor{caption-color}{HTML}{777777}
\usepackage[font={stretch=1.2}, textfont={color=caption-color}, position=top, skip=4mm, labelfont=bf, singlelinecheck=false, justification=raggedright]{caption}
\setcapindent{0em}

%
% blockquote
%
\definecolor{blockquote-border}{RGB}{221,221,221}
\definecolor{blockquote-text}{RGB}{119,119,119}
\usepackage{mdframed}
\newmdenv[rightline=false,bottomline=false,topline=false,linewidth=3pt,linecolor=blockquote-border,skipabove=\parskip]{customblockquote}
\renewenvironment{quote}{\begin{customblockquote}\list{}{\rightmargin=0em\leftmargin=0em}%
\item\relax\color{blockquote-text}\ignorespaces}{\unskip\unskip\endlist\end{customblockquote}}

%
% heading color
%
\definecolor{heading-color}{RGB}{40,40,40}
\addtokomafont{section}{\color{heading-color}}
% When using the classes report, scrreprt, book,
% scrbook or memoir, uncomment the following line.
%\addtokomafont{chapter}{\color{heading-color}}

%
% variables for title and author
%
\usepackage{titling}
\title{}
\author{Chuwen}

%
% tables
%

%
% remove paragraph indention
%
\setlength{\parindent}{0pt}
\setlength{\parskip}{6pt plus 2pt minus 1pt}
\setlength{\emergencystretch}{3em}  % prevent overfull lines

%
%
% Listings
%
%


%
% header and footer
%
\usepackage{fancyhdr}

\fancypagestyle{eisvogel-header-footer}{
  \fancyhead{}
  \fancyfoot{}
  \lhead[\today]{}
  \chead[]{}
  \rhead[]{\today}
  \lfoot[\thepage]{Chuwen}
  \cfoot[]{}
  \rfoot[Chuwen]{\thepage}
  \renewcommand{\headrulewidth}{0.4pt}
  \renewcommand{\footrulewidth}{0.4pt}
}
\pagestyle{eisvogel-header-footer}

%%
%% end added
%%

\begin{document}

%%
%% begin titlepage
%%

%%
%% end titlepage
%%



{
\setcounter{tocdepth}{3}
\tableofcontents
}
this file is dedicated to some thoughts on stochastic properties。

We want to show the same algorithm works for both dynamic and static
models.

\hypertarget{stochastic}{%
  \section{Stochastic}\label{stochastic}}

\hypertarget{dynamic-multistage-model}{%
  \subsection{Dynamic / Multistage model}\label{dynamic-multistage-model}}

We now consider the optimization model under uncertainty. Suppose demand
\(\mathbfit{d}\) is random with respect to some unknown distribution
\(f \in \mathcal F\). Similarly, we wish to solve stochastic model that
minimizes expected summation of shortages and surpluses deviated from
the demand unfold within a finite horizon.

Fruitful research has been done in the field of stochastic programming.
Traditionally, stochastic programming approaches solve the expected
objective that might be too optimistic. (see \ldots) Robust
optimization, in contrast, optimizes a worst-case objective subject to
the ambiguity set (see \ldots). Recently, the distributionally robust
methods (see \ldots) provide a paradigm to minimize the worst-case risk
\ldots{}

The multistage or dynamic models are known to be intractable, where at
each stage decision is made after the realization of uncertain events.
SAA\ldots{} Furthermore, linear decision rules (LDR), see @, and
Bertsimas et al. (2019) provides detailed analysis on LDR named after
\emph{adaptive distributionally robust optimization}.

We notice the stochastic version inherits the property that it could be
decomposed into a set of independent subproblems by relaxing linking
constraints. The idea regarding Lagrangian relaxation to dynamic
optimization models is not new. Hawkins (2003) develops the theory of
Lagrangian relaxation on so-called \emph{weakly coupled Markov decision
  process} with applications to queueing networks, supply-chain management
problems, and multiarmed bandits, et cetera. It provides analysis of
both infinite and finite horizon versions of the problem. Adelman and
Mersereau (2008) later contributes to the bound and optimality gap for
both Lagrangian and linear programming relaxations.

We use boldface notation to denote random variables and corresponding
decision variables.\\
Let \(\Xi_d\) be the support for random variable \(\mathbfit{d}\). Let
\(\mathbfit y = \left[\mathbfit m,\mathbfit s \right]\) be the
variable under uncertainty. Since \(\mathbfit y\) sufficiently
represents the state at period \(t\), we can write the multistage
optimization model using dynamic programming equations.

Define \(V_{t}\) is the optimal value with \(t\) periods to go. Consider
the following multistage stochastic optimization problem:

\[z = \min_{\small{\mathbfit\delta^-_t, \mathbfit\delta^+_t, \mathbfit u_{it}, \mathbfit x_{it}}}
  \mathbb{E}_f \left[ \sum_t^{|T|}h \cdot \mathbfit \delta^-_t + b \cdot \mathbfit \delta^+_t \right ]\]

While the decisions are taken under conditions:
\[\sum_i \mathbfit u_{it} - \mathbfit\delta^-_t + \mathbfit\delta^+_t = \mathbfit{d}_t, \quad \\
  \mathbfit s_{it}, \mathbfit u_{it}, \mathbfit x_{it}\in \Omega_i\]

Now limit the scope to the finite horizon. We are interested in the
expected value with known initial state \(\mathbfit y_0\): \[
  z_T(y_0) = \mathbb{E}_f ()
\]

Similar to the deterministic problem, the Bellman iteration can be
written as:

\[V_{t}(\mathbfit y, \mathbfit{d}_t) =
  \min_{\small{\mathbfit\delta^-_t, \mathbfit\delta^+_t, \mathbfit u_{it}, \mathbfit x_{it}}}
  h \cdot \mathbfit \delta^-_t + b \cdot \mathbfit \delta^+_t + \mathbb E_{f} \left [ V_{t-1}(\mathbfit y', \mathbfit{d}'_{t-1}) \big | \mathbfit y, \mathbfit{d}_t\right]\]

We now investigate the Lagrangian relaxation. The analysis is similar to
existing results in Adelman and Mersereau (2008), Hawkins (2003). The
difference lies in the fact that we do not enforce \(\lambda_t\) to be
identical across the stages \(t = 1, ..., |T|\), which is necessary for
infinite dimensional problems.

\textbf{Proposition 1.} Lagrangian relaxation provides a lower bound for
any multiplier \(\lambda = (\lambda_1, ..., \lambda_{|T|})\) such that
\(\lambda_t \in [-b, h],\; \forall t=1,..., |T|\).
\[V_{t}(\mathbfit y, \mathbfit d_t) \ge -\lambda_t \mathbfit d_t + \sum_{i\in I} V_{it}(\mathbfit y_i, \mathbfit d_t)\]

Where \(V_{it}\) is the optimal equation for each \(i\)

\[V_{it}(\mathbfit y, \mathbfit d_t) = \mathbfit u_{it} \lambda_t + \mathbb E_{f} \left [ V_{i,t-1}(\mathbfit y', \mathbfit{d}'_t) \big | \mathbfit y, \mathbfit{d}_t \right]\]

\textbf{PF.} Relax binding constraints, since any feasible solution is
the solution to the relaxed problem, we have:

\[\begin{aligned}
    V_{t}(\mathbfit y, \mathbfit d_t) & \ge
    \min_{\small{\mathbfit\delta^-_t, \mathbfit\delta^+_t, \mathbfit u_{it}, \mathbfit x_{it}}}
    (h - \lambda_t )\cdot \mathbfit \delta^-_t + (b + \lambda_t) \cdot \mathbfit \delta^+_t
    + \sum_i \mathbfit{u}_{it} \lambda_t - \lambda_t \mathbfit d_t
    + \mathbb E_{f} \left [ V_{t-1}(\mathbfit y', \mathbfit{d}'_t) \big | \mathbfit y, \mathbfit{d}_t \right ] \\
  \end{aligned}\]

The RHS is unbounded unless \(\lambda_t \in [-b, h]\), we have:

\[\begin{aligned}
    V_{t}(\mathbfit y, \mathbfit d_t) & \ge
    \min_{\small{\mathbfit u_{it}, \mathbfit x_{it}}}
    \sum_i \mathbfit{u}_{it} \lambda_t - \lambda_t \mathbfit d_t
    + \mathbb E_{f} \left [ V_{t-1}(\mathbfit y', \mathbfit{d}'_t) \big | \mathbfit y, \mathbfit{d}_t\right]             \\
                                      & = - \lambda_t \mathbfit d_t + \sum_{i\in I} V_{it}(\mathbfit y_i, \mathbfit d_t)
  \end{aligned}\]

\emph{The last line can be verified by induction similar to} Hawkins
(2003). This completes the proof. \(\quad\blacksquare\)

\textbf{Proposition 2.} subgradient of \(\lambda\).

\hypertarget{static-distributionally-robust-model}{%
  \subsection{Static Distributionally Robust
    Model}\label{static-distributionally-robust-model}}

The DRO/SP model, the goal is to minimize worst-case expected
unsatisfied demand and surplus (idle) flights

\[\begin{aligned}
                  & \min \max_{f\in \mathscr F}\mathbb E_f  \left[e^\top( b \cdot\mathbfit{\delta^+}  + h\cdot \mathbfit \delta^-)\right]                                 \\
    \mathbf{s.t.} &                                                                                                                                                       \\
                  & \mathbfit{U} ^\top e + \mathbfit \delta^+ - \mathbfit \delta^-  = \mathbfit d                                         & \forall \mathbfit d \in \Xi_d \\
                  & \mathbfit U_{(i,\cdot)}, \mathbfit X_{(i,\cdot)}, \mathbfit S_{(i,\cdot)} \in \Omega_i                                & \forall i\in I
  \end{aligned}\]

Same relaxation scheme can be used on the DRO models:

\begin{itemize}
  \tightlist
  \item
        Mean-variance, in Delage and Ye (2010).
  \item
        Likelihood, in Wang et al. (2016)
\end{itemize}

\hypertarget{moment-uncertainty}{%
  \subsubsection{Moment Uncertainty}\label{moment-uncertainty}}

With moment uncertainty for
\(\mathbfit d: \mathbb{E}(\mathbfit d) = \mu_0, ...\), in Delage and
Ye (2010). The DRO model is equivalent to the following problem:

\[\begin{aligned}
    \min_{\mathbfit{U}, \mathbfit{Q}, \mathbfit{\beta}, r, s} & \left(\gamma_{2} \mathbfit{\Sigma}_{0}-\mathbfit{\mu}_{0} \mathbfit{\mu}_{0}^{\top}\right) \bullet \mathbfit{Q}+r+\left(\mathbfit{\Sigma}_{0} \bullet \mathbfit{P}\right)-2 \mathbfit{\mu}_{0}^{\top} \mathbfit{p} + \gamma_{1} s                                 \\
    \mathbf { s.t. }                                          &                                                                                                                                                                                                                                                                   \\
                                                              & \mathbfit{Q} \succeq 0, \mathbfit{\beta} \in \mathbb{R}^{|T|},
    \begin{bmatrix}
      \mathbfit{P}      & \mathbfit{p} \\
      \mathbfit{p}^\top & s
    \end{bmatrix} \succeq 0, \;
    \mathbfit \beta = 2 (\mathbfit p + \mathbfit{Q\mu}_0)                                                                                                                                                                                                                                                                         \\
                                                              & \mathbfit{U} ^\top e + \mathbfit \delta^+ - \mathbfit \delta^-  = \mathbfit d                                                                                                                                                     & \forall \mathbfit d \in \Xi_d \\
                                                              & \mathbfit{d}^{\top} \mathbfit{Q} \mathbfit{d} -\mathbfit{d^\top\beta} + r \ge e^\top( b \cdot\mathbfit{\delta^+}  + h\cdot \mathbfit \delta^-)                                                                                    & \forall \mathbfit d \in \Xi_d \\
                                                              & \mathbfit X_{(i,\cdot)}, \mathbfit U_{(i,\cdot)}, \mathbfit S_{(i,\cdot)} \in \Omega_i                                                                                                                                            & \forall i\in I
  \end{aligned}\]

semi-infinite constraints are equivalent to (substitute
\(\mathbfit\delta^- = \mathbfit u + \mathbfit \delta^+ - \mathbfit d\),we
have immediately)

\[\begin{bmatrix}
    \mathbfit Q                  & (he- \mathbfit \beta)/2                                            \\
    (he- \mathbfit \beta)^\top/2 & r - (h+b)e^\top \mathbfit \delta^+ - h e^\top   \mathbfit U^\top e
  \end{bmatrix} \succeq 0\]

wrap up:

\[\begin{aligned}
    \min_{\mathbfit{x}, \mathbfit{Q}, \mathbfit{\beta}, r, s} & \left(\gamma_{2} \mathbfit{\Sigma}_{0}-\mathbfit{\mu}_{0} \mathbfit{\mu}_{0}^{\top}\right) \bullet \mathbfit{Q}+r+\left(\mathbfit{\Sigma}_{0} \bullet \mathbfit{P}\right)-2 \mathbfit{\mu}_{0}^{\top} \mathbfit{p} + \gamma_{1} s                  \\
    \mathbf { s.t. }                                          &                                                                                                                                                                                                                                                    \\
                                                              & \mathbfit{Q} \succeq 0, \mathbfit{\beta} \in \mathbb{R}^{|T|},
    \begin{bmatrix}
      \mathbfit{P}      & \mathbfit{p} \\
      \mathbfit{p}^\top & s
    \end{bmatrix} \succeq 0, \;
    \mathbfit \beta = 2 (\mathbfit p + \mathbfit{Q\mu}_0)                                                                                                                                                                                                                                                          \\
                                                              & \begin{bmatrix}
      \mathbfit Q                  & (he- \mathbfit \beta)/2                                            \\
      (he- \mathbfit \beta)^\top/2 & r - (h+b)e^\top \mathbfit \delta^+ - h e^\top   \mathbfit U^\top e
    \end{bmatrix} \succeq 0                                                                                                                                                                                                                \\
                                                              & \mathbfit X_{(i,\cdot)}, \mathbfit U_{(i,\cdot)}, \mathbfit S_{(i,\cdot)} \in \Omega_i                                                                                                                                            & \forall i\in I
  \end{aligned}\]

is this too complex?

\hypertarget{finite-support-and-likelihood-robust}{%
  \subsubsection{Finite support and likelihood
    robust}\label{finite-support-and-likelihood-robust}}

The problem is, let
\(\mathbfit Q = [\mathbfit{q}^1, ..., \mathbfit{q}^N]\)

\[\begin{aligned}
     & \max_{\beta, \theta, \Omega_i, \forall i} \theta + \beta \gamma +  \beta N - \underbrace{\beta \mathbf N^\top \log(\frac{\beta \mathbf N}{\mathbfit{Q} e-\theta \mathbf 1})}_{\mathcal D_{KL}(\beta \mathbf N | \mathbfit{Q} e-\theta \mathbf 1)}                         \\
    \textbf {s.t.}                                                                                                                                                                                                                                                               \\
     & \beta \ge 0                                                                                                                                                                                                                                                               \\
     & \mathbfit{Q} e \ge \theta \mathbf 1                                                                                                                                                                                                                                       \\
     & \mathbfit{U} ^\top e + \mathbfit \delta^+ - \mathbfit \delta^-  = \mathbfit d^n                                                                                                                                                                   & \forall n = 1, ..., N \\
     & \mathbfit x_{(i,\cdot)}, \mathbfit u_{(i,\cdot)}, \mathbfit s_{(i,\cdot)} \in \Omega_i                                                                                                                                                            & \forall i\in I
  \end{aligned}\]

\hypertarget{reference}{%
  \section*{Reference}\label{reference}}
\addcontentsline{toc}{section}{Reference}

\hypertarget{refs}{}
\begin{cslreferences}
  \leavevmode\hypertarget{ref-adelman_relaxations_2008}{}%
  Adelman D, Mersereau AJ (2008) Relaxations of weakly coupled stochastic
  dynamic programs. \emph{Operations Research} 56(3):712--727.

  \leavevmode\hypertarget{ref-bertsimas_adaptive_2019}{}%
  Bertsimas D, Sim M, Zhang M (2019) Adaptive distributionally robust
  optimization. \emph{Management Science}.

  \leavevmode\hypertarget{ref-delage_distributionally_2010}{}%
  Delage E, Ye Y (2010) Distributionally robust optimization under moment
  uncertainty with application to data-driven problems. \emph{Operations
    Research} 58(3).

  \leavevmode\hypertarget{ref-hawkins_langrangian_2003}{}%
  Hawkins JT (2003) \emph{A langrangian decomposition approach to weakly
    coupled dynamic optimization problems and its applications}. PhD thesis.
  (Massachusetts Institute of Technology).

  \leavevmode\hypertarget{ref-wang_likelihood_2016}{}%
  Wang Z, Glynn PW, Ye Y (2016) Likelihood robust optimization for
  data-driven problems. \emph{Computational Management Science}
  13(2):241--261.
\end{cslreferences}

\end{document}

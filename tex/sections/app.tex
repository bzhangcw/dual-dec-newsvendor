\documentclass[../main]{subfiles}

\begin{document}

\section{Applications}\label{sec:app}

\subsection{Fleet Engine Maintenance Problem}\label{sec:fmp}

In the Fleet Engine Maintenance Problem (FMP), recurrent maintenance is needed for each airplane to ensure safety.

At each time \(t \in T\) there is a demand of quantity \(d_t \ge 0\) associated with withdraw cost \(b \ge 0\).
If the size of the fleet at current time is greater than demand, then it incurs the idle cost \(h > 0\).
Each airplane \(i \in I\) deteriorates with rate \(\alpha_i\) and there is a lower bound \(L\) on the lifespan representing the current condition. If the airplane approaches to the worst-allowed-condition then it cannot be assigned to any flights.
A maintenance plan should be scheduled to improve the current condition for plane \(i\) by rate \(\beta_i\). Once scheduled, a plane comes back after \(\tau\) time periods.

The goal is to minimize the total cost by uncovered demand and surplus flights. We summarize the notation as follows:

\begin{quote}
  Notation
\end{quote}

\begin{itemize}
  \tightlist
  \item
        \(I, T\) - set of plane, time periods, respectively
  \item
        \(b, h\) - demand withdraw and plane idle cost, respectively
  \item
        \(\tau\) - lead time for maintenance
\end{itemize}

We first assume the demand is deterministic.

\begin{itemize}
  \tightlist
  \item
        \(d_t\) - demand, number of planes needed at time \(t\)
\end{itemize}

We make a plan to define work and maintenance schedules.

\begin{quote}
  Decision
\end{quote}

\begin{itemize}
  \tightlist
  \item
        \(x_{it}\) - 0 - 1 variable, 1 if plane \(i\) starts a maintenance at
        time \(t\)
  \item
        \(u_{it}\) - 0 - 1 variable, 1 if plane is working at time \(t\)
  \item
        \(s_{it} \ge 0\) - the lifespan of plane \(i\) at time \(t\)
\end{itemize}


The objective can be written in the Newsvendor style:

\begin{equation}\label{eq:fmp.newsvendor_obj}
  \min_{u,x,s} b \cdot (d_t - \sum_i u_{it})_+ + h \cdot  ( \sum_i u_{it} - d_t)_+
\end{equation}

Alternatively, we use the following objective function with
\(\delta_t, \epsilon_t\) indicating unsatisfied demand and surplus,
respectively.

\begin{align}
  \label{eq:fmp.obj}  f =    & \min_{x_{it}, u_{it}, \delta_t, \epsilon_t} \sum_t (b\cdot  \delta_t + h \cdot \epsilon_t)                                                       \\
  \nonumber \mathbf{s.t.}    &                                                                                                                                                  \\
  \label{eq:fmp.demand}      & \sum_i u_{it} + \delta_t - \epsilon_t = d_t                                                & \forall t \in T                                     \\
  \label{eq:fmp.lifespan}    & s_{i, t+1} =  s_{i t}  - \alpha_i  u_{it} + \beta_i  x_{i, t- \tau}                        & \forall i \in I, t \in T                            \\
  \label{eq:fmp.work_or_not} & x_{it} +  u_{i, t} \le 1                                                                   & \forall i \in I, t \in T                            \\
  \label{eq:fmp.leadtime}    & x_{it} + x_{i\rho} + u_{i, \rho} \le 1                                                     & \forall i \in I,  t\in T, \rho = t + 1, ..., t+\tau \\
  \label{eq:fmp.lifespan_lb} & s_{i t} \ge L                                                                              & \forall i \in I, t \in T
\end{align}

The objective function \eqref{eq:fmp.obj} and the binding constraint \eqref{eq:fmp.demand} follow the same routine for Newsvendor objective, cf. \eqref{eq:primal}.
The last four sets of constraint describe the non-overlapping requirements during a maintenance for each \(i\).
\eqref{eq:fmp.lifespan} tracks the lifespan at each period \(t\), \eqref{eq:fmp.work_or_not} describes the utility status of each plane.
The non-overlapping requirements for working and maintenance is indicated in \eqref{eq:fmp.leadtime}.
We summarize \eqref{eq:fmp.lifespan} - \eqref{eq:fmp.lifespan_lb} as \(\Omega_i\).

Let \(U, X, S \in \mathbb R^{|I|\times |T|}_+\) be the matrix of
\(u_{it}, x_{it}\) and \(s_{it}\), \(U_{(i,.)}\) be the \(i\)th row of
\(U\). Let \(\delta, \epsilon\) be the vector of
\(\delta_t, \epsilon_t\), respectively. It allows a more compact
formulation.

\begin{align}
                                & \min_{U, X, S}  e^\mathsf{T} (b\cdot  \delta + h \cdot \epsilon)                   \\
  \nonumber \mathbf{s.t.}       &                                                                                    \\
  \label{eq:fmp.compact.demand} & U^\mathsf{T} e + \delta - \epsilon = d                           & \forall t \in T \\
                                & X_{(i,\cdot)}, U_{(i,\cdot)}, S_{(i,\cdot)} \in \Omega_i         & \forall i \in I
\end{align}


% If we allow a tolerance \(\epsilon \ge 0\) on subgradient method, the
% worst overall complexity is
% \(O\left(\frac{1}{\epsilon^2}\cdot\tau\cdot|I|\cdot|T|^3\right)\).


\subsubsection{Dual Optimization}\label{sec:fmp.dual_opt}

Similar to previous analysis, the Lagrangian is introduced by relaxing the equality constraint \eqref{eq:fmp.compact.demand},
so we have:

\[\begin{aligned}
    \phi(\lambda) & = - \sum_t \lambda_t d_t + \min_{\delta_t, \epsilon_t, U} \sum_t \left [ (b + \lambda_t) \cdot \delta_t + (h-\lambda_t)\cdot \epsilon_t \right ] + \sum_i \sum_t\lambda_t u_{it} \\
  \end{aligned}\]

It reduces to a set of low dimensional minimization problems for each \(i \in I\):

\begin{equation}\label{eq:fmp.dual_phi}
  \begin{aligned}
    \phi(\lambda)   & = - \sum_t \lambda_t d_t  + \min_{U}\sum_i \sum_t\lambda_t u_{it} \\
    \mathbf {s.t. } &                                                                   \\
                    & X_{(i,\cdot)}, U_{(i,\cdot)}, S_{(i,\cdot)} \in \Omega_i          \\
                    & -b \le \lambda_t \le h
  \end{aligned}
\end{equation}


Lagrange multipliers is updated by a subgradient method \eqref{eq:main_subgrad}, and the primal solution is computed by the Recovery Algorithm \eqref{eq:recovery} using the averaged scheme.

Next we provide analysis on properties of the subproblem.

\subsubsection{Subproblem}\label{sec:fmp.subproblem-for-each-plane}

In the dual search process, one should solve a set of subproblems
\(\forall i\in I\) with respect to \(\lambda\) defined as follows:

\begin{equation}\begin{aligned}
    \min_{\Omega_i} \sum_t \lambda_t \cdot u_{i,t}
  \end{aligned}\end{equation}

The model seeks to minimize total cost while keeping the
lifespan safely away from the lower bound \(L\). We solve this by
dynamic programming.

Define state: \(y_t = \left[m_t,s_t \right]^\mathsf{T}\), where \(m_t\)
denotes current working status is the remaining lifespan.
At each period \(t\) we decide whether the plane \(i\) is idle, working, or starting a maintenance, i.e.:

\[(u_t, x_t) \in \left\{(1, 0), (0,0), (0, 1)\right\}\]

We have the Bellman equation:
\begin{equation}\label{eq:fmp.dp.deterministic}
  V_t(u_t, x_t | y_t) = \lambda_t \cdot u_t + \min_{u,x} V_{t-1}(u_{t-1}, x_{t-1} | y_{t-1})
\end{equation}

Complexity: let \(s_0\) be the initial lifespan and finite time horizon
be \(|T|\), we notice the states for remaining maintenance waiting time
is finite, \(m_t \in \{0, 1, ..., \tau\}\).

Let total number of possible periods to initiate a maintenance be
\(n_1\), and working periods be \(n_2\). If we ignore lower bound \(L\)
on \(s\), total number of possible values of \(s\) is bounded above:
\(|s| \le (|T| + 1)(\frac{1}{2}|T| + 1)\)
since \(n_1 + n_2 \le |T|\). For each subproblem we have at most 3
actions, thus we conclude this problem can be solved by dynamic
programming in polynomial time, the complexity is:
\(O\left(\tau\cdot|T|^3 \right)\)




% \begin{itemize}
%   \tightlist
%   \item
%         *compute \(\min c ^\mathsf{T} | x - x^\star|\) where \(x^\star\) is the
%         (possibly) fractional solution achieving the best bound, using DP.
% \end{itemize}

% still working on this.
% \subsection{Fleet Engine Maintenance Problem: Extensions}\label{sec.sfmp}


% We now assume the demand is stochastic with some distribution \(f\in \mathscr F\).
% We use boldface notation to denote random variables and corresponding
% decision variables.

% \subsubsection{Distributionally Robust Fleet Engine Maintenance Problem}\label{sec.lro_fmp}


% \[\begin{aligned}
%                   & \min \max_{f\in \mathscr F}\mathbb E_f  \left[e^\mathsf{T}( b \cdot\mathbfit{\delta}  + h\cdot \mathbfit\epsilon)\right]                                 \\
%     \mathbf{s.t.} &                                                                                                                                                          \\
%                   & \mathbfit{U} ^\mathsf{T} e + \mathbfit \delta - \mathbfit\epsilon  = \mathbfit d                                         & \forall \mathbfit d \in \Xi_d \\
%                   & \mathbfit U_{(i,\cdot)}, \mathbfit X_{(i,\cdot)}, \mathbfit S_{(i,\cdot)} \in \Omega_i                                   & \forall i\in I
%   \end{aligned}\]

% Let \(\mathbfit Q = [\mathbfit{q}^1, ..., \mathbfit{q}^N]\)

% \[\begin{aligned}
%      & \max_{\beta, \theta, \Omega_i, \forall i} \theta + \beta \gamma +  \beta N - \underbrace{\beta \mathbf N^\mathsf{T} \log(\frac{\beta \mathbf N}{\mathbfit{Q} e-\theta \mathbf 1})}_{\mathcal D_{KL}(\beta \mathbf N | \mathbfit{Q} e-\theta \mathbf 1)}                         \\
%     \textbf {s.t.}                                                                                                                                                                                                                                                                     \\
%      & \beta \ge 0                                                                                                                                                                                                                                                                     \\
%      & \mathbfit{Q} e \ge \theta \mathbf 1                                                                                                                                                                                                                                             \\
%      & \mathbfit{U} ^\mathsf{T} e + \mathbfit \delta - \mathbfit\epsilon  = \mathbfit d^n                                                                                                                                                                      & \forall n = 1, ..., N \\
%      & \mathbfit X_{(i,\cdot)}, \mathbfit U_{(i,\cdot)}, \mathbfit S_{(i,\cdot)} \in \Omega_i                                                                                                                                                                  & \forall i\in I
%   \end{aligned}\]

% This problem can be solved by the same methods mentioned in Section \ref{sec:dual}, cf. \eqref{eq:main_subgrad}, \eqref{eq:recovery}.

% \subsubsection{Dynamic Fleet Engine Maintenance Problem}\label{sec.dynamic_fmp}

% Let \(\Xi_d\) be the support for random variable \(\mathbfit{d}\). Let
% \(\mathbfit y = \left[\mathbfit m,\mathbfit s \right]\) be the
% variable under uncertainty. Since \(\mathbfit y\) sufficiently
% represents the state at period \(t\), we can write the multistage
% optimization model using dynamic programming equations.

% Define \(V_{t}\) is the optimal value with \(t\) periods to go. Consider
% the following multistage stochastic optimization problem:

% \begin{equation}\label{eq:dynamic_fmp.model}
%   z_T(y) = \min_{\mathbfit\epsilon_t, \mathbfit\delta_t, \mathbfit u_{it}, \mathbfit x_{it}}
%   \mathbb{E}_f \left[ \sum_{t=1}^{|T|}h \cdot \mathbfit \epsilon_t + b \cdot \mathbfit \delta_t  | \mathbfit y \right]
% \end{equation}

% Decision are made under restrictions:

% \begin{align}
%    & \sum_i \mathbfit u_{it} - \mathbfit\epsilon_t + \mathbfit\delta_t = \mathbfit{d}_t, \quad \\
%    & \mathbfit s_{it}, \mathbfit u_{it}, \mathbfit x_{it}\in \Omega_i
% \end{align}

% We limit the scope to finite horizon. We are interested in the
% expected value with known initial state \(\mathbfit y_0\).
% Similar to the deterministic problem, the Bellman iteration can be
% written as:

% \begin{equation}\label{eq:dynamic_fmp.main_iteration}
%   V_{t}(\mathbfit y, \mathbfit{d}_t) =
%   \min_{\small{\mathbfit\epsilon_t, \mathbfit\delta_t, \mathbfit u_{it}, \mathbfit x_{it}}}
%   h \cdot \mathbfit \epsilon_t + b \cdot \mathbfit \delta_t +
%   \mathbb E_{f} \left [ V_{t-1}(\mathbfit y', \mathbfit{d}'_{t-1}) \big | \mathbfit y, \mathbfit{d}_t\right]
% \end{equation}

% We now investigate the Lagrangian relaxation. The analysis is similar to
% existing results in \cite{adelman_relaxations_2008}, \cite{hawkins_langrangian_2003}.
% % The difference lies in the fact that we do not enforce \(\lambda_t\) to be identical across the stages \(t = 1, ..., |T|\), which is necessary for infinite dimensional problems.

% \begin{lemma}
%   Lagrangian relaxation provides a lower bound for
%   any multiplier \(\lambda = (\lambda_1, ..., \lambda_{|T|})\) such that
%   \(\lambda_t \in [-b, h],\; \forall t=1,..., |T|\).
%   \begin{equation}\label{eq:d}
%     V_{t}(\mathbfit y, \mathbfit d_t) \ge
%     \phi_t(\mathbfit y, \mathbfit d_t) \stackrel{\mathsf{def}}{=}
%     - \lambda_t \mathbfit d_t + \sum_{i\in I} V_{it}(\mathbfit y_i, \mathbfit d_t)
%   \end{equation}

%   Where \(V_{it}\) is the optimal equation for each \(i\)

%   \begin{equation}\label{eq:dynamic_fmp.sub_iteration}
%     V_{it}(\mathbfit y, \mathbfit d_t) = \min_{\mathbfit u_{it}}\mathbfit u_{it} \lambda_t + \mathbb E_{f} \left [ V_{i,t-1}(\mathbfit y', \mathbfit{d}'_t) \big | \mathbfit y, \mathbfit{d}_t \right]
%   \end{equation}

% \end{lemma}
% \begin{proof}
%   Relax binding constraints, since any feasible solution is
%   the solution to the relaxed problem, we have:

%   \[\begin{aligned}
%       V_{t}(\mathbfit y, \mathbfit d_t) & \ge
%       \min_{\small{\mathbfit\epsilon_t, \mathbfit\delta_t, \mathbfit u_{it}, \mathbfit x_{it}}}
%       (h - \lambda_t )\cdot \mathbfit \epsilon_t + (b + \lambda_t) \cdot \mathbfit \delta_t
%       + \sum_i \mathbfit{u}_{it} \lambda_t - \lambda_t \mathbfit d_t
%       + \mathbb E_{f} \left [ V_{t-1}(\mathbfit y', \mathbfit{d}'_t) \big | \mathbfit y, \mathbfit{d}_t \right ] \\
%     \end{aligned}\]

%   The RHS is unbounded unless \(\lambda_t \in [-b, h]\), we have:

%   \[\begin{aligned}
%       V_{t}(\mathbfit y, \mathbfit d_t) & \ge
%       \min_{\small{\mathbfit u_{it}, \mathbfit x_{it}}}
%       \sum_i \mathbfit{u}_{it} \lambda_t - \lambda_t \mathbfit d_t
%       + \mathbb E_{f} \left [ V_{t-1}(\mathbfit y', \mathbfit{d}'_t) \big | \mathbfit y, \mathbfit{d}_t\right]             \\
%                                         & = - \lambda_t \mathbfit d_t + \sum_{i\in I} V_{it}(\mathbfit y_i, \mathbfit d_t)
%     \end{aligned}\]

%   The last line can be verified by induction similar to \cite{hawkins_langrangian_2003}.
%   This completes the proof.
% \end{proof}



\subsubsection{Numerical Experiments}\label{sec.fmp:numerical-experiments}

In this section, we report numerical results to
demonstrate the efficiency and effectiveness of our proposed algorithms
for solving the FMP. We parallelize the
subproblems to available cores solved by dynamic programming.
We summarize all deterministic test cases in Table \ref{tab:comp_repair_cases}.
\end{document}
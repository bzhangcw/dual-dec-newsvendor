\documentclass[../main]{subfiles}

\begin{document}

\hypertarget{Applications}{%
  \section{Applications}\label{applications}}

\subsection{Flight Maintenance Problem}\label{app:fmp}

Consider a fleet where recurrent maintenance is needed for each airplane to ensure safety.
At each time \(t \in T\) there is a demand of quantity \(d_t \ge 0\) associated with withdraw cost \(b \ge 0\).
If the size of the fleet at current time is greater than demand, then it incurs the idle cost \(h > 0\).
Each airplane \(i \in I\) deteriorates with rate \(\alpha_i\) and there is a lower bound \(L\) on the lifespan representing the current condition. If the airplane approaches to the worst-allowed-condition then it cannot be assigned to any flights.
A maintenance plan should be scheduled to improve the current condition for plane \(i\) by rate \(\beta_i\). Once scheduled, a plane comes back after \(\tau\) time periods.

The goal is to minimize the total cost by uncovered demand and surplus flights. We summarize the notation as follows:

\begin{quote}
  Notation
\end{quote}

\begin{itemize}
  \tightlist
  \item
        \(I, T\) - set of plane, time periods, respectively
  \item
        \(b, h\) - demand withdraw and plane idle cost, respectively
  \item
        \(\tau\) - lead time for maintenance
\end{itemize}

% The demand is stochastic with some distribution \(f\in \mathscr F\).
We first assume the demand is deterministic.

\begin{itemize}
  \tightlist
  \item
        \(\mathbfit d_t\) - demand, number of planes needed at time \(t\)
\end{itemize}

We make a plan to define work and maintenance schedules.

\begin{quote}
  Decision
\end{quote}

\begin{itemize}
  \tightlist
  \item
        \(x_{it}\) - 0 - 1 variable, 1 if plane \(i\) starts a maintenance at
        time \(t\)
  \item
        \(u_{it}\) - 0 - 1 variable, 1 if plane is working at time \(t\)
  \item
        \(s_{it} \ge 0\) - the lifespan of plane \(i\) at time \(t\)
\end{itemize}


The objective can be written in the Newsvendor style:

\[\min_{u,x,s} b \cdot (d_t - \sum_i u_{it})_+ + h \cdot  ( \sum_i u_{it} - d_t)_+ \]

Alternatively, we use the following objective function with
\(\delta_t, \epsilon_t\) indicating unsatisfied demand and surplus,
respectively. Let \(z\) be the objective function

\[\begin{aligned}
    z =           & \min_{x_{it}, u_{it}, \delta_t, \epsilon_t} \sum_t (b\cdot  \delta_t + h \cdot \epsilon_t)                                                       \\
    \mathbf{s.t.} &                                                                                                                                                  \\
                  & \sum_i u_{it} + \delta_t - \epsilon_t = d_t                                                & \forall t \in T                                     \\
                  & s_{i, t+1} =  s_{i t}  - \alpha_i  u_{it} + \beta_i  x_{i, t- \tau}                        & \forall i \in I, t \in T                            \\
                  & x_{it} +  u_{i, t} \le 1                                                                   & \forall i \in I, t \in T                            \\
                  & x_{it} + x_{i\rho} + u_{i, \rho} \le 1                                                     & \forall i \in I,  t\in T, \rho = t + 1, ..., t+\tau \\
                  & s_{i t} \ge L                                                                              & \forall i \in I, t \in T
  \end{aligned}\]

We define the last four sets of constraint as \(\Omega_i\), which
describe the non-overlapping requirements during a maintenance for each
\(i\).

Let \(U, X, S \in \mathbb R^{|I|\times |T|}_+\) be the matrix of
\(u_{it}, x_{it}\) and \(s_{it}\), \(U_{(i,.)}\) be the \(i\)th row of
\(U\). Let \(\delta, \epsilon\) be the vector of
\(\delta_t, \epsilon_t\), respectively. It allows a more compact
formulation.

\[\begin{aligned}
                  & \min_{U, X, S}  e^\top (b\cdot  \delta + h \cdot \epsilon)                   \\
    \mathbf{s.t.} &                                                                              \\
                  & U^\top e + \delta - \epsilon = d                           & \forall t \in T \\
                  & X_{(i,\cdot)}, U_{(i,\cdot)}, S_{(i,\cdot)} \in \Omega_i   & \forall i \in I
  \end{aligned}\]


% If we allow a tolerance \(\epsilon \ge 0\) on subgradient method, the
% worst overall complexity is
% \(O\left(\frac{1}{\epsilon^2}\cdot\tau\cdot|I|\cdot|T|^3\right)\).

\hypertarget{lagrangian-relaxation}{%
  \subsubsection{Lagrangian Relaxation}\label{lagrangian-relaxation}}

The Lagrangian is introduced by relaxing the equality constraint, so we
have:

\[\begin{aligned}
    \phi(\lambda) & = - \sum_t \lambda_t d_t + \min_{\delta_t, \epsilon_t, U} \sum_t \left [ (b + \lambda_t) \cdot \delta_t + (h-\lambda_t)\cdot \epsilon_t \right ] + \sum_i \sum_t\lambda_t u_{it} \\
  \end{aligned}\]

\(\phi(\lambda)(\lambda)\) is unbounded unless
\(-b \le \lambda_t \le h\), it reduces to a set of low dimensional
minimization problems for each \(i\):

\[\begin{aligned}
    \phi(\lambda)   & = - \sum_t \lambda_t d_t  + \min_{U}\sum_i \sum_t\lambda_t u_{it} \\
    \mathbf {s.t. } &                                                                   \\
                    & X_{(i,\cdot)}, U_{(i,\cdot)}, S_{(i,\cdot)} \in \Omega_i          \\
                    & -b \le \lambda_t \le h
  \end{aligned}\]

Next we provide analysis on properties of the subproblem.

\hypertarget{subproblem-for-each-plane}{%
  \subsubsection{Subproblem for each
    plane}\label{subproblem-for-each-plane}}

In the dual search process, one should solve a set of subproblems
\(\forall i\in I\) defined as follows:

\[\begin{aligned}
    \min_{\Omega_i} \sum_t \lambda_t \cdot u_{i,t}
  \end{aligned}\]

The model describes a problem to minimize total cost while keeping the
lifespan safely away from the lower bound \(L\). We solve this by
dynamic programming.

Define state: \(y_t = \left[m_t,s_t \right]^\top\), where \(m_t\)
\textbf{denotes the remaining time of the undergoing maintenance}.
\(s_t\) is the remaining lifespan. At each period \(t\) we decide
whether the plane \(i\) is idle or waiting (for the maintenance),
working, or starting a maintenance, i.e.:

\[(u_t, x_t) \in \left\{(1, 0), (0,0), (0, 1)\right\}\]

We have the Bellman equation:
\[V_n(u_t, x_t | m_t, s_t) = \lambda_t \cdot u_t + \min_{u,x} V_{n-1}(...)\]

Complexity: let \(s_0\) be the initial lifespan and finite time horizon
be \(|T|\), we notice the states for remaining maintenance waiting time
is finite, \(m_t \in \{0, 1, ..., \tau\}\).

Let total number of possible periods to initiate a maintenance be
\(n_1\), and working periods be \(n_2\). If we ignore lower bound \(L\)
on \(s\), total number of possible values of \(s\) is bounded above:
\(|s| = \sum_i^{|T|}\sum_j^{|T| - i} 1=(|T| + 1)(\frac{1}{2}|T| + 1)\)
since \(n_1 + n_2 \le |T|\). For each subproblem we have at most 3
actions, thus we conclude this problem can be solved by dynamic
programming in polynomial time, the complexity is:
\(O\left(\tau\cdot|T|^3 \right)\)

\hypertarget{subgradient-method}{%
  \subsubsection{Subgradient Method}\label{subgradient-method}}

Lagrange multipliers is updated by a subgradient method. (volume
algorithm, etc.)

\textbf{The Volume Algorithm HERE}

Notice:

\begin{itemize}
  \item
        At iteration \(k\), suppose
        \(-b \le \lambda^k_t \le h, \forall t \in T\), we use dynamic
        programming to solve the relaxed minimization problem, then the
        (integral) solution \(( X^k, S^k, U^k)\) is also feasible for the
        original problem (compute \(\delta, \epsilon\) accordingly). The
        primal value \(z^k\) is the upper bound for optimal solution
        \(z^\star\): \(z^k\ge z^\star\).
  \item
        In the volume algorithm, we consider the convex combination \(\bar X\)
        of past iterations \(\{X^1, ..., X^k\}\). We update
        \(\bar X \leftarrow \alpha X^k + (1-\alpha) \bar X\). It's easy to
        verify \(\bar z \ge z^\star \ge z_{\textsf{LD}}^k\), where \(\bar z\)
        is the primal objective value for \(\bar X\) and \(z_{\textsf{LD}}^k\)
        is the dual value for \(X^k\). By the termination criterion
        \(|\bar z - z_{\textsf{LD}}^k| \le \epsilon_z\) for some small value
        \(\epsilon_z >0\), we conclude the \(\bar z\) converges to the optimal
        value \(z^\star\).
  \item
        While \(\bar z \to z^\star\), there is no guarantee for the solution
        \(\bar X, \bar U\) being integral via the volume algorithm;
        \(\bar X, \bar U\) is feasible only to the linear relaxation.
  \item
        Remark:

        \begin{itemize}
          \tightlist
          \item
                The projection for dual variables is simple since there is only a
                box constraint. More computation would be needed if we use the
                minimax objective function, i.e.,
                \(q \ge h\cdot (U^\top e - d), q\ge b\cdot (d-U^\top e)\), in which
                case two set of multipliers are needed, say \(\lambda, \mu \ge 0\),
                and the projection should be done onto:
                \[\{(\lambda,\mu)|\lambda +\mu \le 1\}\]
        \end{itemize}
\end{itemize}

\hypertarget{rounding}{%
  \subsubsection{Rounding}\label{rounding}}

\begin{itemize}
  \tightlist
  \item
        *compute \(\min c ^\top | x - x^\star|\) where \(x^\star\) is the
        (possibly) fractional solution achieving the best bound, using DP.
\end{itemize}

still working on this.

\hypertarget{numerical-experiments}{%
  \subsubsection{Numerical Experiments}\label{numerical-experiments}}

In this section, In this section, we report numerical results to
demonstrate the efficiency and effectiveness of our proposed algorithms
for solving the repair problem (\textbf{ref here}). We parallelize the
subproblems to available cores solved by dynamic programming.
\end{document}